\documentclass{article}
\title{Title of the project}
\date{release date: \today\\version 0: skeleton}
\author{Matteo Secco, Mohammad Rahbari}
\usepackage{enumitem}
\newcommand{\enum}[1]{\texttt{#1.\arabic*}}
\begin{document}
\pagenumbering{gobble}
\maketitle
\newpage
\tableofcontents
\pagebreak
\pagenumbering{arabic}
\section{Introduction}
	\subsection{Purpose} \textit{here we	 include	 the	 goals of the project}
\paragraph{}The required system, called SafeStreets, is a distributed system to allow the citizens to signal parking violations to the competent autorities.
\paragraph{}The system must allow the citizen to submit pictures of the violation, attaching data such as date, time and position. The user will have to specify the type of the violation when sending these data. 
\paragraph{}When reciving such data, the system must store them, toghether with the plate of the car that performed the violation (elictated from the picture the citizen sent), the informations about the violation itself (in particular the type of violation), and the name of the street where the violation occourred (which can be retrieved by the positioning information the user sent).
\paragraph{}In addition, the application must allow both autorities and citizens to analyze the stored data, for example highlighting streets or plates with most violations registered. Different levels of security can be offered.
\paragraph{}Finally, the application must be able to automatically generate traffic tickets for people commiting violations, if and only if it can be verified that the data concerning it has not been altered. In this case, SafeStreets could use this informations to build additional statistics.
\paragraph{Goal list}
\begin{enumerate}[label=\enum{G}]
	\item  Allow citizens to notify parking violations in real time
	\item \label{G_allData}Allow citizens to provide all the needed data about violation, in particular infraction type, picture, date, time and position
	\item Prevent the autorities to have to manually address parking tickets
	\item \label{G_discardAlteredNot} Ensure no ticket is given if the notification's data has been modified somehow
	\item \label{G_respectPermissions} Ensure no ticket is given if the plate of the car that committed the infringment owns a permission for that infringiment
	\item \label{G_ticketsEmitted} Every notification not covered by \ref{G_discardAlteredNot} or \ref{G_respectPermissions} will always generate a ticket
	\item \label{G_statistics}Allow both citizens and authorities retrive informations about previous violations and released tickets, possibly in an aggregated form 
\end{enumerate}

	\subsection{Scope} \textit{here we include an analysis of the world and of the shared phenomena}\\
	The world where the system must fit is an everday city, with focus on the traffic of moterized veichles.\\
	The events the system aims to influence are the parking of motorized veichles,  in particular the ones considered infractions.\\
	In the context of the system, when any user notices an illegal parking, he/she may notify the system and provide any needed informations to the competent authorities. In particular, the notification is composed by a picture of the infraction, a timestamp (date and time), the geographical location of the infraction and the type of infraction wich is to be notified. Some of these informations can be gathered automatically from the user's device.\\
	In addiction, the user may interrogate the system to gather aggregated informations about the locations with more violation incidence, and the cars which committed more violations. 
	\begin{list}{-}{World phenomena}
		\item item
	\end{list}
	\begin{list}{-}{Shared phenomena}
		\item hello
	\end{list}
	\subsection{Definitions, Acronyms,Abbreviations} \label{definitions}
		\paragraph{Person:}A person in the real world. Every Citizen is a person, generally an Authorithy is not
		\paragraph{User:}A person, an organization or a system which somehow uses SafeStreets
		\paragraph{Citizen (cit):} This term will be used to denote every \underline{user} not owning particular privileges or permissions. A citizen is only allowed to notify violations and see some aggregated data
		\paragraph{Authority (auth):} This term will denote every \underline{user} (phisical or digital) having privileged access to the stored data. An example of Authority is the Local Police.
		\paragraph{Notification:} 
			\begin{list}{-}{Represents a set of data submitted by any user composed by:}
				\item A picture of a parking infraction
				\item A timestamp of when the notification occourred, containing date and time (may be gathered automatically by the citizen's device)
				\item A geographical position of where the infraction occurred (may be gathered automatically by the citizen's device)
				\item The type of infraction notified
			\end{list}
		\paragraph{Car:}The word car will be used to issue every mothorized vehicle
		\paragraph{Plate:}Identifies a \underline{car}
		\paragraph{Permisison:}A document released by a verified authority, granting to a car the permission to park in a set of reserved parkings (ex: permission to park on parking reserved for disabled people). 
		
%	\subsection{Revision history}
%	\subsection{Reference documents}
%	\subsection{Document Structure}
\section{Overall descriprion}
	\subsection{Product perspective} \textit{here we include further details on the shared phenomena and a domain model (class diagrams and statecharts)}
	\subsection{Product functions}\textit{here we include anything that is relevant to clarify their needs}
	\subsection{Assumptions, dependencies and constraints}
	\paragraph{Assumptions:}
	\begin{enumerate}[label=\enum{A}]
		\item \label{A_disjPlates} Different cars always have different plates
		\item \label{A_Single plate}Each car exactly has 1 plate
		\item \label{A_singleOwner}No car is owned by more than 1 person
		\item \label{A_accessiblePermissions}If an authority is enabled to give tickets, that authorty will have access to every permission
			\subitem In particular, if the auth is evaluating a ticket for plate p, it will be able to check every permission granted to p
	\end{enumerate}
	\paragraph{Constraints:}
	\begin{enumerate}[label=\enum{C}]
		\item A constraint
		\item Another constraint
	\end{enumerate}
\section{Specific Requirements} \textit{Here we include more details on all the aspects in Section 2 if they can be used for the development team}
	\subsection{External Interface Requirements}
		\subsubsection{User Interfaces}
		\subsubsection{Hardware Interfaces}
		\subsubsection{Software Interfaces}
		\subsubsection{Communication Interfaces}
	\subsection{Functional Requirements} \textit{Definition of use case diagrams, usa cases and associated sequence/activity diagrams, and mapping on requirements}
		\subsubsection{Scenarios}
			\paragraph{The man, the street and the monstertruck}
				Mario is walking down the street when he comes to a cross. He would like to \textbf{cross the cross} staying on the pedestrian lines, but a monstertruck decided to park over there. So Mario needs to walk around the truck and through the street (cars are driving fast), but at least he can take a little revenge by notifying the police via SafeStreets and knowing the crazy-minded driver will pay what he deserves! (like some fines)
			\paragraph{Prevent is better than healing}Luigi is late for work. He cannot find any parking, and is thinking about parking on the bycicle line. He's about to do that, when he suddently remembers that nowaday anyone could signal his infractions with SafeStreets. So he decides to keep searching for a proper parking, and granny Maria can take her daily ride on her bycicle!
	\subsection{Performance requirements}
	\subsection{Design Constraints}
		\subsubsection{Standard compliance}
		\subsubsection{Hardware limitations}
		\subsubsection{Any other constraint}
	\subsection{Software system attributes}
		\subsubsection{Reliability}
		\subsubsection{Availability}
		\subsubsection{Security}
		\subsubsection{Maintainability}
		\subsubsection{Portability}
\section{Formal analysis using alloy} \textit{This section should include a brief presentation of the main objectives driving the formal modeling activity, as well as a description of the model itself, what can be proved with it, and why what is proved is important given the problem at hand. To show the soundness and correctness of the model, this section can show some worlds obtained by running it, and/or the results of the checks performed on meaningful assertions}
	\paragraph{}The signatures are omitted in the document as they reflect the entities 				desctibes in section \ref{definitions}, plus some "enumerations". The goals we were able to define in Alloy are \ref{G_discardAlteredNot},\ref{G_respectPermissions} and \ref{G_ticketsEmitted}.
	There was no need to check \ref{G_statistics} and \ref{G_allData} as they can be satisfied 	by an equal requirement
	\paragraph{}The assumptions we were able to define are \ref{A_disjPlates}, \ref{A_Single plate}, \ref{A_singleOwner}, \ref{A_accessiblePermissions}. Please note not all of them has been expressed by facts as some was already expressed by multiplicity in the signatures.
\section{Effort spent} \textit{In this section you will include information about the number of hours each group member has worked for this document}
	\paragraph{Matteo Secco} 
		\begin{list}{-}{}
			\item October 15: created the skeleton. 1h
			\item October 16: Introduction and Scope. 30m
			\item October 16: Definitions. 15m.
			\item October 23: Scenarios and goals. toghether. 1h
			\item October 24: Update goals, assumptions, alloy skeleton. 2h
		\end{list}
	\paragraph{Rahbari}
		\begin{list}{-}{}
			\item October 23: Scenarios and goals. toghether. 1h
		\end{list}
\section{References}
\end{document}
